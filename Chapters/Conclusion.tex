\chapter{Conclusions}
\label{chpr:conclusion}
The aim of this thesis is to give an introduction to the Schnorr signature algorithm, starting from the mathematics and the cryptography behind the scheme, and present some of its amazing applications to Bitcoin, detailing the benefits and the improvements that would arise from its deployment. We started with a brief but thorough description of the mathematical structures (Chapter \ref{chpr:math}) and cryptographic primitives (Chapter \ref{chpr:ecc}) that underpin digital signature schemes based on elliptic curve cryptography. In Chapter \ref{chpr:dss} we presented both ECDSA and Schnorr algorithm, respectively the one actually implemented in Bitcoin and the one that is under development. We compared the two schemes, investigating ECDSA lacks and Schnorr benefits, that ranged from security to efficiency. In particular we focused on the linearity property, that turned out to be the key for the higher level construction presented in Chapter \ref{chpr:application}.
\\
We have seen how to traduce utilities already implemented in Bitcoin in terms of Schnorr signatures: multi-signature schemes are implemented through MuSig (Section \ref{musig}), whose main advantage is to recover key aggregation; threshold signatures can be deployed through the protocols presented in Section \ref{threshold}, that makes them indistinguishable from a single signature; the last application we studied has been adaptor signature and its benefits to cross-chain atomic swaps and to the Lightning Network.

\bigskip
\noindent
The immediate benefits that Schnorr would bring to Bitcoin are improved efficiency (smaller signatures, batch validation, cross-input aggregation) and privacy (multi-signatures and threshold signatures would be indistinguishable from a single signature), leading also to an enhancement in fungibility. All this applications would be possible in a straightforward way after the introduction of Schnorr, that could be brought to Bitcoin through a soft-fork\footnote{Improvements in the protocol have to be made without consesus split.}: the fact that Schnorr is superior to ECDSA in every aspect hopefully will ease the process.

\bigskip
\noindent
The last thing we would like to point out is that, by no means, the applications presented in the present work are the unique benefits that Schnorr could bring to Bitcoin. More complex ideas take the names of Taproot \cite{Taproot} and Graftroot \cite{Graftroot}, and are built on top of the concept of MAST and Pay-to-Contract: through these constructions it would be possible, in the cooperative case, to hide completely the redeem script, presenting a single signature (no matter how complex the script is). For how soft forks need to be implemented after SegWit (i.e. with an upgrade of the version number), there is incentive to develop as many innovations as possible altogether (the presence of too many version numbers with little differences would constitute a lack of privacy): for this reason, it is probable that Schnorr will come to life accompanied by Taproot. 
\\
Hopefully, we have convinced the reader that Schnorr (and Bitcoin!) is worth being studied, providing also the tools to properly understand further features and innovations other than the ones presented. Moreover, we hope that you are now motivated not only to delve deeper in the technical side of Bitcoin, but also to approach it from other sides, to fully appreciate its disruptiveness and make yourself an idea of what Bitcoin is and which possibilities it hides.