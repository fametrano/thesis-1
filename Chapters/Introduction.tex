\chapter{Introduction}
\label{chpr:intro}
\section{Problem under analysis}
Bitcoin (\cite{BTC}) is a decentralized network whose aim is to provide a peer-to-peer electronic payment system. In recent years it has astonished people who had the opportunity to study it, with an increasing number of supporters and detractors worldwide: it has been called a bubble, a Ponzi scheme, it has been labelled as sound money and store of value. But labels can be dangerous, and excessive labeling is usually useless: for this reason the present work does not aim at settling the dispute. Bitcoin is hard to understand, being at the crossroad of various disciplines, going from networking theory to cryptography, from economics to game theory. Among this multitude, mathematics and cryptography are perhaps the easiest way to approach Bitcoin, being difficult to misinterpret these two disciplines: of course there are engineering tradeoffs and debates around them, but the technical foundations are sound.
\\
For this reason, the present work wants to give a brief but satisfactory introduction to the mathematical structures and the cryptographic primitives behind Bitcoin, focusing in particular on the role played by the digital signature scheme. We will outline the problems presented by ECDSA, the scheme actually implemented, and show how they could be solved by Schnorr, a scheme that might be introduced in the protocol in the coming years. On top of that, we will delve into the technicalities of higher level constructions enabled by Schnorr, that will help in solving some of the key problems of Bitcoin, namely privacy and fungibility\footnote{Fungibility is the property of a good or a commodity whose individual units are interchangeable.}. Bitcoin was intended to be anonymous, but it is not: it is pseudonymous, in the sense that it is possible to link transactions to a unique identity through blockchain analysis. If this digital identity is unveiled, connecting to a real identity, that person could be persecuted for his involvement in Bitcoin (e.g. in authoritarian regime). The lack of privacy contributes to the lack of fungibility, allowing to distinguish between different bitcoins\footnote{Typically Bitcoin identifies the protocol, while bitcoins are the coins transferred using the protocol.}. These problems affects heavily the monetary use case and in general have to be properly addressed: their improvement will be the \textit{leitmotiv} of this present work.

\bigskip
\noindent
The present thesis has been written during the author's fellowship at the Digital Gold Institute, a research and development center focused on teaching, consulting, and advising about scarcity in digital domain (Bitcoin and crypto-assets) and the underlying blockchain technology. The author has actively contributed to the development of the Python library that can be found at \url{https://github.com/dginst/BitcoinBlockchainTechnology}: for this reason, some optimization procedures are presented throughout the thesis.

\bigskip

\bigskip

\section{Thesis structure}
In Chapter \ref{chpr:math} we will start presenting the mathematical foundations of the cryptography underpinning the Bitcoin protocol: in particular we will recall the concepts of group and field (Section \ref{groupsfields}), we will introduce the general idea of elliptic curve (EC, Section \ref{ec}) and show how to induce a group structure on an EC through a proper definition of the addition operation (Section \ref{grouplaw}).

\bigskip
\noindent
In Chapter \ref{chpr:ecc} we will give an overview of the public key cryptography based on elliptic curves: thanks to the mathematical introduction of Chapter \ref{chpr:math} we will be able to understand what an elliptic curve over a finite field is (Section \ref{ecoverff}); then we will discuss elliptic curves' parameters selection and validation (Section \ref{ecparam}), we will present the basic idea behind asymmetric cryptography based on EC (Section \ref{keypairs}) and we will show how to improve the core operation of asymmetric elliptic curve cryptography (ECC), scalar multiplication, through a proper change of coordinates (Section \ref{jac}). As a practical use case the section will be concluded with the presentation of the Bitcoin curve (secp256k1 in Section \ref{btccurve}). The chapter will be concluded by the description of the hard problem at the basis of elliptic curves' widespread adoption in recent years, the so called discrete logarithm problem (DLP in Section \ref{dlp}).

\bigskip
\noindent
Chapter \ref{chpr:dss} will deal with ECDSA (Section \ref{ecdsa}) and Schnorr (Section \ref{schnorr}). Throughout the chapter many topics will be touched, among which the issues of ECDSA, Schnorr's solutions and multi-signature schemes.

\bigskip
\noindent
Finally in Chapter \ref{chpr:application} we will delve in the technicalities of Schnorr's applications: we will study a multi-signature scheme that recovers key aggregation (MuSig in Section \ref{musig}), a threshold signature scheme (Section \ref{threshold}) and the benefits arising from the use of adaptor signatures (Section \ref{adaptor}) for cross-chain atomic swaps (Section \ref{atomic}) and for the Lightning Network (Section \ref{ln}).

\bigskip
\noindent
Chapter \ref{chpr:conclusion} will conclude the thesis with a summary in which we will draw interesting conclusions.
\bigskip

\bigskip

\section{Notation}
\begin{itemize}
\item Prime numbers: the lowercase letter $p$ is used to represent an odd prime number;
\item Fields: in Chapter \ref{chpr:math} for a general field the letter $K$ is used, while the finite fields of order $q = p^k$, where $p$ is an odd prime and $k$ an integer, are represented as $\mathbb{F}_q$;
\item Elliptic curves: in general an elliptic curve over a field $K$ is denoted by $E(K)$, which represents the set of points satisfying the generalized Weierstrass equation with coefficients in $K$ plus the point at infinity. From Chapter \ref{chpr:ecc} onward, we will deal with EC over finite fields: in this case lowercase letters are used to denote scalars, while the uppercase equivalent denotes the linked EC point, e.g. $qG = Q = (x_Q, y_Q)$ ($G$ is reserved to the generator of the group). Whenever a second generator is needed we use the capital letter $H$: this does not constitute a conflict with the cofactor $h$ of the group since typically the two generators are NUMS (\textit{nothing up my sleeves}), meaning that we do not know the discrete logarithm of one with respect to the other and viceversa;
\item Elliptic curves' key pair: the pair of private and public key is denoted as $\{q, Q\}$, where $Q = qG$. Whenever a second point is needed we use the couple $\{r, R\}$; if more key pairs are needed subscripts are used, e.g. $q_1G = Q_1 = (x_1, y_1)$, $q_2G = Q_2 = (x_2, y_2)$ and $q_3G = Q_3 = (x_3, y_3)$. Notice that, for the sake of clarity, also the coordinate representation is adapted.
\item Algorithms:
    \begin{itemize}
		\item $||$ refers to byte array concatenation;
		\item $a \gets b$ refers to the operation of assignment;
		\item $z \xleftarrow{\text{\$}} Z$ denotes uniform sampling from the set $Z$ and assignment to $z$;
		\item The function $\text{bytes}(x)$, where $x$ is an integer, returns the byte encoding of $x$;
		\item The function $\text{bytes}(Q)$, where $Q$ is an EC point, returns $bytes(0\text{x}02 + (y_Q \& 1))$ $ || \ bytes(x_Q)$\footnote{An EC point can be expressed in compressed or uncompressed form, as suggested in \cite{RefWork:2}: the compressed form requires the $x$ coordinate and one additional byte to make explicit the $y$ coordinate, while the uncompressed form requires both the coordinates.};
		\item The function $\text{int}(x)$, where $x$ is a byte array, returns the unsigned integer whose byte encoding is $x$;
		\item The function $\text{hash}(x)$, where $x$ is a byte array, returns the hash of $x$. In particular, when dealing with the Schnorr signature, it returns the 32 byte SHA-256 of $x$;
		\item The function $\text{jacobi}(x)$, where $x$ is an integer, returns the Jacobi symbol $\left(\frac{x}{p}\right)$. In general we have (gcd here stands for greatest common divisor): $$\left(\frac{x}{p}\right)= \begin{cases} 0, & \text{if gcd}(x, p) \neq 1 \\ \pm 1, & \mbox{if gcd}(x, p) = 1 \end{cases}$$
		Moreover we have that if $\left(\frac{x}{p}\right) = - 1$ then $x$ is not a quadratic residue modulo $p$ (i.e. it has not a square root modulo $p$) and that if $x$ is a quadratic residue modulo $p$ and $\text{gcd}(x, p) = 1$, then $\left(\frac{x}{p}\right) = 1$. However, unlike the Legendre symbol, if $\left(\frac{x}{p}\right) = 1$ then $x$ may or may not be a quadratic residue modulo $p$. Fortunately enough, since $p$ is an odd prime we have that the Jacobi symbol is equal to the Legendre symbol. Thus we can check only whether $\left(\frac{x}{p}\right) = 1$ to assess if $x$ is or is not a quadratic residue modulo $p$.
		\\
		Moreover, by Euler's criterion we have that $\left(\frac{x}{p}\right) = x^{\frac{p - 1}{2}} \ (\text{mod} \ p)$, so that we have an efficient way to compute the Jacobi symbol.
	\end{itemize}	
\end{itemize}
